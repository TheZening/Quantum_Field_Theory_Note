%==============================
%==============================
\chapter{Appendix 1: 重要恒等式以及证明}
%==============================
%==============================

%====================
%====================
\section{Pauli代数}
%====================
%====================
\begin{table}[htbp!]
	\centering
	\begin{tabular}{c|cccc}
	& $\sigma_0$ & $\sigma_x$ & $\sigma_y$ & $\sigma_z$\\
	\hline
	$\sigma_0$ & $\sigma_0$ & $\sigma_x$ & $\sigma_y$ & $\sigma_z$\\
	$\sigma_x$ & $\sigma_x$ & $\sigma_0$ & $\mathrm{i}\,\sigma_z$ & $-\mathrm{i}\,\sigma_y$\\
	$\sigma_y$ & $\sigma_y$ & $-\mathrm{i}\,\sigma_z$ & $\sigma_0$ & $\mathrm{i}\,\sigma_x$\\
	$\sigma_z$ & $\sigma_z$ & $\mathrm{i}\,\sigma_y$ & $-\mathrm{i}\,\sigma_x$ & $\sigma_0$
	\end{tabular}
  \label{Tab: Appendix_1_Pauli_Matrix_Multiplication}
	\caption{Pauli矩阵相乘图}
\end{table}

%====================
%====================
\section{Barker-Campell-Hausdorff公式}
%====================
%====================
如果$\hat{A}$和$\hat{B}$是两个有界算符, 我们有:
\begin{equation}
	e^{\hat{A}} e^{\hat{B}} = e^{\hat{Z}}
\end{equation}
其中$\hat{Z}$是$\hat{A}$和$\hat{B}$的Barker-Campell-Hausdorff公式:
\begin{equation}
	\hat{Z} = \hat{A} + \hat{B} + \frac{1}{2} [\hat{A}, \hat{B}] + \frac{1}{12} [\hat{A}, [\hat{A}, \hat{B}]] + \frac{1}{12} [\hat{B}, [\hat{A}, \hat{B}]] - \frac{1}{24} [\hat{A}, [\hat{B}, [\hat{A}, \hat{B}]]] + \cdots
\end{equation}
这个公式可以用来计算两个算符的指数的乘积. 要加到无穷阶才能收敛, 我们实际上使用的是几项近似, 很少算到后面的阶数.
实际上, 证明这个公式是非常复杂的.
我们最关心的一种特殊情况就是满足如下的条件:
\begin{equation}
	[\hat{A}, [\hat{A}, \hat{B}]] = 0 \quad \text{同时} \quad [\hat{B}, [\hat{A}, \hat{B}]] = 0
\end{equation}
我们才能获得一个更简单的公式:
\begin{equation}
	e^{\hat{A}} e^{\hat{B}} = e^{\hat{A} + \hat{B} + \frac{1}{2} [\hat{A}, \hat{B}]}
\end{equation}